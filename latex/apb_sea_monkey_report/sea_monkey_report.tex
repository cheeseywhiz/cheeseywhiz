% Spencer Todd
% September 2018, AP Biology, Bollone

\documentclass[12pt]{article}

% \usepackage[long]{sea_monkey_report}
\usepackage{sea_monkey_report}

\begin{document}
\title{Sea Monkey Lab Report}{By Spencer Todd}\double

\bigheader{Question}\\
How will the sea monkey egg hatch rate be affected in potassium chloride (KCl) solution?\double

\bigheader{Knowledge Probe}\\
Sea monkeys are salt water creatures, so they require salt to survive. Fish also use potassium to function.\footnote{\textit{Using
Salt to Transport Live Fish}, \url{www2.ca.uky.edu/wkrec/SALTTRANS.PDF}, Accessed 8 September 2018.} As sodium and postassium are
both alkali metals, they have similar chemical properties.\double

\bigheader{Prediction}\\
Since there is no salt in the solution, the sea monkey eggs will have a decreased hatch rate.\double

\bigheader{Investigation Plan}\\
\lilheader{Equipment}
\begin{itemize}
    \item 2 petri dishes
    \item Double sided tape
    \item 2 graduated cylinders
    \item Distilled water
    \item Scale
    \item 2 weigh boats
    \item Salt
    \item Potassium chloride
    \item Q Tip
    \item 50 to 100 sea monkey eggs
    \item Stereoscope
\end{itemize}

\lilheader{Procedure}
\begin{enumerate}
    \item Label the petri dishes \control\ and \test.
    \item Stick $1\inches$ of double sided tape in the center of each petri dish.
    \item Fill each graduated cyninder with $100\mL$ of distilled water.
    \item With the scale and a weigh boat, prepare $1.5\grams$ of salt and pour it into the first graduated cylinder, then shake
    the cylinder to dissolve all of the salt.
    \item With the scale and another weigh boat, prepare $1.5\grams$ of potassium chloride and pour it into the second graduated
    cylinder, then shake the cylinder to dissolve all of the potassium chloride.
    \item With the Q Tip, divide the sea monkey eggs between both petri dishes, sticking them on the tape.
    \item With the stereoscope, count and record the number of intact eggs in each dish.
    \item Pour $30\mL$ of the salt solution in to the \control\ dish, then put the top on the dish.
    \item Pour $30\mL$ of the potassium chloride solution in to the \test\ dish, then put the top on the dish.
    \item Wait 24 hours.
    \item With the stereoscope, count and record the number of swimming sea monkeys in each dish.
\end{enumerate}\double

\bigheader{Observations}
\begin{center}
\begin{tabular}{l|rr}
Dish & Total eggs & Number of swimmers \\
\hline
Control & 40 & 14 \\
Test & 84 & 0 \\
\end{tabular}
\end{center}\double

\bigheader{Data Analysis}\\
We can use the \chisq\ test with the null hypothesis that the choice between the control or test was independent of the amount of
swimming sea monkeys. We choose the significance level $\alpha$ to be $0.05$. We can arrange our observations into the following
two way table.
%
\begin{center}
\begin{tabular}{lrr|r}
& n swim & n not swim & \\
Control & 14 & 26 & 40 \\
Test & 0 & 84 & 84 \\\hline
& 14 & 110 & 124
\end{tabular}
\end{center}
%
We can determine the expected values by dividing the product of a cell's row and column sum by the table sum.
%
\begin{center}
\begin{tabular}{lrr|r}
& n swim & n not swim & \\
Control & $\frac{40\cdot{14}}{124}=4.52$ & $\frac{40\cdot{110}}{124}=35.5$ & 40 \\
Test & $\frac{84\cdot{14}}{124}=9.48$ & $\frac{84\cdot{110}}{124}=74.5$ & 84 \\\hline
& 14 & 110 & 124
\end{tabular}
\end{center}
%
Then, we can apply the \chisq\ formula, $\frac{\paren{o-e}^2}{e}$, to corresponding cells in the observation and expected value
tables.
%
\begin{center}
\begin{tabular}{lrr|r}
& n swim & n not swim & \\
Control & 19.9 & 2.53 & 22.5 \\
Test & 9.48 & 1.21 & 10.7 \\\hline
& 29.4 & 3.74 & 33.1
\end{tabular}
\end{center}
%
The statistic \chisq\ shall not be greater than $3.84$ at $p=0.05$ for 1 degree of freedom. With $\chi^2=33.1$, we can reject the
null hypothesis. Thus, taking the test correlates with the amount of swimming sea monkeys.\double

\bigheader{Explanation}\\
\lilheader{Claim}\\
Potassium chloride solution will decrease the hatch rate of sea monkey eggs.\double

\lilheader{Evidence}\\
In the potassium chloride test, $0$ out of $84$ eggs managed to hatch in the same time that it took $14$ out of $40$ eggs to hatch
in salt water. With $\chi^2=33.1$ between the control and the test, the test had significantly fewer hatched sea monkeys.\double

\lilheader{Reasoning}\\
The evidence supports the claim because when potassium chloride is dissolved in water instead of salt, the amount of swimming sea
monkeys decreases. This makes sense, because sea monkeys need salt to survive. Unlike humans, who get salt from eating food, sea
monkeys get salt from swimming in salt water. Removing the salt deprives the sea monkeys of a vital resource.\double

The evidence is not reliable, because there was an insufficient amount of test samples. There were only 2 dishes, while there
should have been at least 7 dishes to satisfy the \chisq\ test requirements. Also, it was assumed that $1.5\grams$ of potassium
chloride yielded 1.5\% concentration in the solution, as is the case with salt. This test could be repeated with a more rigorous
way to determine concentration. Furthermore, all of our sea monkey eggs came from the same source. Choosing a random sample from
the global sea monkey population would be more statistically robust.\double

\bigheader{Evaluation}\\
I would like to know how the sea monkeys would respond if salt and potassium chloride, or other salts, were disolved in water at
the same time. Furthurmore, I could test which concentrations and combinations of salts that the sea monkeys like best.\double

\bigheader{Application}\\
This experiment can be used to determine how well sea monkeys would survive in a water environment that has different
concentrations of salts, possibly such as a deep sea floor.

\end{document}
