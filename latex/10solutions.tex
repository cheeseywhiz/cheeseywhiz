\documentclass[10pt]{article}

\usepackage[margin=1in]{geometry}
\usepackage[protrusion=true,expansion=true]{microtype}
\usepackage{lmodern}
\usepackage{amsmath}
\DeclareMathOperator{\sech}{sech}
\DeclareMathOperator{\invcos}{cos^{-1}}

\linespread{1.6}
\setlength\parindent{0em}

\newcommand{\bfit}[2]{\textbf{#1}\ \textit{#2}}
\renewcommand{\title}[2]{\huge\bfit{#1}{#2}\normalsize}
\newcommand{\double}[0]{\par\null\par}
\renewcommand{\section}[2]{\double\LARGE\bfit{#1}{\##2}\normalsize\\}
\newcommand{\ddx}[0]{\frac{d}{dx}}
\renewcommand{\exp}[1]{e^{#1}}
\newcommand{\paren}[1]{\left({#1}\right)}

\let\xint\int
\renewcommand{\int}[2]{\xint{#1}\,d#2}

\begin{document}
\fontfamily{lmr}\selectfont
\title{10 detailed solutions}{by Spencer Todd}
\section{6.1}{39}
Find $\paren{f^{-1}}'\paren{a}$ of $f\paren{x}=3x^3+4x^2+6x+5$, $a=5$.\double

Theorem 7 in Section 6.1 establishes that if $f$ is one-to-one, its inverse is
$f^{-1}$, and $f'\paren{f^{-1}\paren{a}}\neq 0$, then
%
\begin{equation*}
\paren{f^{-1}}'\paren{a}=\frac{1}{f'\paren{f^{-1}\paren{a}}}.
\end{equation*}

To begin, a sketch of the graph of $f$ shows that it is one-to-one.
\vspace{1.5in}

Next, we must find the derivative of $f\paren{x}$, then apply Theorem 7. Additionally, $f\paren{x}=5$ when $x=0$, so
$f^{-1}\paren{5}=0$.
%
\begin{align*}
f'\paren{x}&=9x^2+8x+6 \\
%
\paren{f^{-1}}'\paren{a}&=\frac{1}{f'\paren{f^{-1}\paren{5}}} \\
&=\frac{1}{f'\paren{0}} \\
&=\frac{1}{6}.
\end{align*}

\section{6.8}{29}
Find the limit:
%
\begin{equation*}
\lim_{x\to 0}\frac{\tanh{x}}{\tan{x}}
\end{equation*}\double

Since $\displaystyle\lim_{x\to 0}\tanh{x}=0$ and $\displaystyle\lim_{x\to 0}\tan{x}=0$, this limit is of the
$\frac{0}{0}$ indeterminate form, so we can use L'H\^{o}pital's Rule.
%
\begin{align*}
\lim_{x\to 0}\frac{\tanh{x}}{\tan{x}}&=\lim_{x\to 0}\frac{\ddx\tanh{x}}{\ddx\tan{x}} \\
&=\lim_{x\to 0}\frac{\sech^2 x}{\sec^2 x} \\
&=\frac{1^2}{1^2} \\
&=1.
\end{align*}

\section{7.1}{9}
Evaluate the integral:
%
\begin{equation*}
\int{\invcos{x}}{x}
\end{equation*}\double

Since $\invcos{x}$ has an easy derivative, we can use integration by parts by differentiating the $\invcos{x}$ and
integrating the $dx$.
%
\begin{equation*}
\int{u}{v}=u\,v-\int{v}{u}
\end{equation*}
%
\begin{alignat*}{2}
u&=\invcos{x} &\quad dv&=dx \\
du&=\frac{-dx}{\sqrt{1-x^2}} &\quad v&=x
\end{alignat*}
%
\begin{align*}
\int{\invcos{x}}{x}&=x\invcos{x}-\int{\frac{-x}{\sqrt{1-x^2}}}{x}
\end{align*}
%
We can set everything inside the radical to a new variable, $a$.
%
\begin{align*}
a&=1-x^2 \\
da&=-2x\,dx \\
-x\,dx&=\frac{1}{2}\,da
\end{align*}
%
\begin{align*}
\int{\invcos{x}}{x}&=x\invcos{x}-\xint{\frac{da}{2\sqrt{a}}} \\
&=x\invcos{x}-\sqrt{a}+C \\
&=x\invcos{x}-\sqrt{1-x^2}+C.
\end{align*}

\section{7.2}{1}
Evaluate the integral:
%
\begin{equation*}
\int{\sin^2{x}\cos^3{x}}{x}
\end{equation*}\double

Since the power of $\cos$ is odd, $z=\sin{x}$ will work.
%
\begin{align*}
z&=\sin{x} \\
\frac{dz}{dx}&=\cos{x} \\
dx&=\frac{dz}{\cos{x}}
\end{align*}

We can use a trigonometric identity to handle the $\cos$.
%
\begin{align*}
\sin^2{x}+\cos^2{x}&=1 \\
\cos^2{x}&=1-\sin^2{x} \\
&=1-z^2
\end{align*}

Blocking the expression as follows will highlight our choices of substitutions:
%
\begin{align*}
&\xint\paren{\sin{x}}^2\paren{\cos^2{x}}\paren{\cos{x}\,dx} \\
&\xint{z}^2\paren{1-z^2}\paren{\cos{x}\,\frac{dz}{\cos{x}}} \\
&\int{z^2-z^4}{z} \\
&\frac{z^3}{3}-\frac{z^5}{5}+C \\
&\frac{\sin^3{x}}{3}-\frac{\sin^5{x}}{5}+C.
\end{align*}

\section{9.5}{7}
Solve the differential equation:
%
\begin{equation*}
y'+y=x \tag{1}
\end{equation*}\double
We are given an inhomogenous equation, so we must first solve the corresponding homogenous equation:
%
\begin{align*}
\frac{dy_0}{dx}+y_0&=0 \\
\frac{dy_0}{dx}&=-y_0 \\
\frac{dy_0}{y_0}&=-dx \\
\xint\frac{dy_0}{y_0}&=-\xint{dx} \\
\ln{|y_0|}&=-x+A \\
|y_0|&=\exp{-x+A} \\
y_0&=B\exp{-x}\text{, whereas $B=\pm\exp{A}$}
%
\intertext{Returning to the original equation, we can replace $B$ with a differable function, $u$.}
%
y&=u\,\exp{-x} \tag{2} \\
y'&=u'\exp{-x}-u\,\exp{-x} \tag{3}
%
\intertext{Now, we can plug (2) and (3) in to (1) and solve for $u$. Notice that a nice cancellation occurs.}
%
u'\exp{-x}-u\,\exp{-x}+u\,\exp{-x}&=x \\
u'\exp{-x}&=x \\
u'&=x\,\exp{x} \\
\xint{u}&=\int{x\,\exp{x}}{x}
\end{align*}

We may proceed using integration by parts $\paren{\int{g}{h}=g\,h-\int{h}{g}}$.
%
\begin{alignat*}{2}
g&=x &\quad dh&=\exp{x}\,dx \\
dg&=dx &\quad h&=\exp{x}
\end{alignat*}
%
\begin{align*}
u&=x\,\exp{x}-\int{\exp{x}}{x} \\
&=x\,\exp{x}-\exp{x}+C \tag{4}
\intertext{Finally, we can plug (4) into (2) for our final answer.}
y&=\exp{-x}\paren{x\,\exp{x}-\exp{x}+C} \\
&=x-1+C\exp{-x}
\end{align*}

\end{document}
