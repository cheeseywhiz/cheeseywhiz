% Spencer Todd
% February 2018, AP Physics Nerf Gun Lab Report, Mrs. Berryhill

\documentclass[12pt]{article}

\usepackage[margin=1in]{geometry}
\usepackage[protrusion=true,expansion=true]{microtype}
\usepackage{lmodern}
\usepackage{amsmath}
\usepackage{graphicx}
\graphicspath{{img/}}
\newcommand{\paren}[1]{\left({#1}\right)}
\newcommand{\double}[0]{\par\null\par}

\DeclareMathOperator{\m}{\ m}
\DeclareMathOperator{\ms}{\ m/s}
\DeclareMathOperator{\mssq}{\ m/s^2}
\DeclareMathOperator{\kg}{\ kg}

\linespread{1.6}
\setlength\parindent{0em}

\newcommand{\bigheader}[1]{\LARGE\textbf{#1}\normalsize}
\newcommand{\header}[1]{\large\textbf{#1}\normalsize}
\newcommand{\lilheader}[1]{\textit{#1}}

\begin{document}\fontfamily{lmr}
\bigheader{Nerf Gun Lab Report}

\lilheader{Spencer Todd}

\renewcommand{\abstractname}{Introduction}
\begin{abstract}\normalsize
The purpose of the lab is to determine which approach can more accurately measure the muzzle speed of a Nerf bullet. The two
methods are based on the kinematic equations and the conservation of momentum. The kinematics approach is possible since the
kinematic equations arise due to that velocity is the integral of acceleration, and position is the integral of velocity. The
kinematic equations assume constant acceleration, whereas acceleration due to gravity is constant. The conservation of momentum
approach is possible since momentum is conserved in a collision, and that momentum is mass multiplied by velocity. I predict that
the kinematic equation approach will be more accurate, since the method will depend only on the motion of the one object instead
of two.
\end{abstract}

\double\bigheader{Kinematics Approach}

\header{Investigation Overview}

\lilheader{Setup}

The premise of the investigation is that the bullet is fired horizontally from a raised position then its travel distance is
recorded. A level was used to orient the gun horizontally and the vertical height of the bottom of a bullet was recorded. The
distance to a reference line on the ground was measured and remembered\footnote{Measurements that were remembered were
intermediate measurements that were not necessary for the final calculations.}. After each trial, the bullet's travel distance
was calculated from the distance to the reference line, then it was recorded. The shape of the tip of the bullet was not
significant, although the same type of bullet was used each time. Five trials were performed.

\double\lilheader{A Trial}

Two observers prepare for the launch, waiting to see where the bullet will land. One person placed the handle of the gun on a
stool while another person aided in adjusting the height of the bullet to the predetermined height. Then, the bullet was shot and
the observers measured the distance of travel as described before.

\double\header{Observations}

The vertical height of the bottom of the bullet was 0.765 meters.

\begin{center}
\begin{tabular}{lr|r}
& travel distance (m) & $v_i=\frac{d}{\sqrt{\frac{2h}{g}}}$ (m/s) \\
\cline{2-3}
& 7.695 & 19.475 \\
& 8.790 & 22.246 \\
& 8.740 & 22.120 \\
& 11.100 & 28.092 \\
& 8.965  & 22.689 \\
\hline
Mean & & 22.924 \\
Standard deviation & & 3.153 \\
\end{tabular}
\end{center}

\double\header{Data Analysis}

The horizontal and vertical motion of the bullet:
\begin{align*}
x\paren{t}&=\frac{a}{2}t^2+v_i{t}+x_i   & y\paren{t}&=\frac{a}{2}t^2+v_i{t}+y_i \\
a&=0\mssq                               & a&=-g \\
&                                       & v_i&=0\ms \\
x_i&=0\m                                & y_i&=h \\
x\paren{t}&=v_i{t}                      & y\paren{t}&=-\frac{g}{2}t^2+h.
\end{align*}

The time when the bullet hits the ground:
\begin{align*}
y\paren{t}&=0\m \\
0&=-\frac{g}{2}t^2+h \\
t&=\sqrt{\frac{2h}{g}}.
\end{align*}

The horizontal distance when the bullet hits the ground:
\begin{equation*}
x\paren{\sqrt{\frac{2h}{g}}}=d=v_i\sqrt{\frac{2h}{g}}.
\end{equation*}

Solving for $v_i$ leaves us with only one free variable, $d$, which was determined with the investigation. All other variables are
constant.
\begin{equation*}
v_i=\frac{d}{\sqrt{\frac{2h}{g}}}
\end{equation*}

\includegraphics[width=\textwidth]{kinematics_approach.png}

Notice how we can keep $h$ constant and vary $d$ to find $v_i$. Now, with $h=0.765\m$, $g=9.8\mssq$, and the travel distance as
$d$, we can compute the initial velocity for each trial. For example, the initial velocity for trial 1 is
\begin{align*}
v_i&=\frac{7.695}{\sqrt{\frac{2\cdot0.765}{9.8}}} \\
&=19.475\ms.
\end{align*}

\double\null\par\bigheader{Momentum Approach}

\header{Investigation Overview}

\lilheader{Setup}

The premise of the investigation is that the bullet is fired upwards towards an object of a known mass, then the change in height
of the object is recorded. To begin, the object (a soda can) was connected to the ceiling with a string. The initial height of the
bottom of the object was remembered. A vertical ruler aided in measuring heights throughout the investigation.  The masses of the
object and the bullet were recorded. It was significant that suction cup bullets were used in order for the bullet to stick to the
object after the collision. The same bullet was used each time. Five trials were performed.

\double\lilheader{A Trial}

The gun was fired upwards towards the object. A video of the trial was recorded, which was used to determine the maximum height of
the bottom of the object. The difference between the object's maximum and initial height was recorded. The trial was repeated if
the bullet did not stick to the object.

\double\header{Observations}\double

\begin{center}
\begin{tabular}{l|r}
& Mass (kg) \\ \hline
Object & 0.0136 \\
Bullet & 0.0014
\end{tabular}

\begin{tabular}{lr|r}
& change in height (m) & $v_0=\frac{m+M}{m}\sqrt{2gh}$ (m/s) \\
\cline{2-3}
& 0.060 & 11.619\\
& 0.088 & 14.071 \\
& 0.086 & 13.910 \\
& 0.090 & 14.230 \\
& 0.064 & 12.000 \\
\hline
Mean & & 13.166 \\
Standard deviation & & 1.251 \\
\end{tabular}
\end{center}

\double\header{Data Analysis}

The order of events during the trial can lead us towards the solution.
First, momentum is conserved in the collision.
Then, the kinetic energy resulting from the collision is converted into potential energy.
\begin{align*}
p_0&=p_1 \\
mv_0&=\paren{m+M}v_1 \\
v_1&=\frac{mv_0}{m+M} \\
\null \\
K_1&=U_2 \\
\frac{1}{2}\paren{m+M}v_1^2&=\paren{m+M}gh \\
v_1&=\sqrt{2gh} \\
\frac{mv_0}{m+M}&=\sqrt{2gh} \\
v_0&=\frac{m+M}{m}\sqrt{2gh}
\end{align*}

\begin{center}
\includegraphics[width=\dimexpr\textwidth/2\relax]{momentum_approach.png}
\end{center}

Notice that since $m$ and $M$ are constant, we can vary $h$ to find $v_0$. Now, with $m=0.0014\kg$, $M=0.0136\kg$,
$g=9.8\mssq$, and the change in height as $h$, we can compute the initial velocity for each trial. For example, the initial
velocity for trial 1 is
\begin{align*}
v_0&=\frac{0.0014+0.0136}{0.0014}\sqrt{2\cdot9.8\cdot0.060} \\
&=11.619\ms.
\end{align*}

\double\bigheader{Explanation}

The momentum approach was more accurate at determining the muzzle velocity of a Nerf bullet.\double

The standard deviation is a way to reasonably calculate the precision of data. The standard deviation of the kinetics approach
data was 3.153, while the standard deviation of the momentum approach data was only 1.251.\double

The evidence supports the data, because a smaller standard deviation is the result of more consistent data. We can expect that
future trials would fall closer to the mean in the momentum approach than the kinematics approach. I had predicted that the
motion of the bullet through the air would be more consistent than the motion of the bullet in a collision. In reality, the
bullet was rather inconsistently influenced by air resistance. In the momentum approach, the gun was able to be placed very
close to the soda can, minimizing any outside influences. The force of air resistance, acting in the opposite direction of the
bullet's velocity, accounted for the travel distance inconsistency.\double

This overall investigation was a fair test, because we were able to accurately collect the data measurements. Redundant
observers and video replays made for undisputable accuracy.\double

\double\bigheader{Evaluation}

\header{Sources of Error}
\begin{itemize}
\item The kinematic equations assume that the bullet flew essentially in a vacuum. In reality, air resistance may have slowed the
bullet in the horizontal direction, or it may have made it fall slower, like a ski jumper.
\item We have assumed that the bullet and can's kinetic energy was perfectly transferred into potential energy. In reality, the
can became off balance and likely transferred energy to many other forms.
\item We measured the change in height of the bottom of the can. This was just an arbitrary choice that was easy to measure.
Determining the change in height of the center of mass of the system may have yielded more accurate data.
\item We have assumed that the string had no effect. In physics, we assume that objects are held by "light" strings, which have
no mass and consistent tension. In reality, the string was massive and slightly elastic, so it had uneven tension throughout its
length.
\item We have assumed that the bullet did not slow down before colliding with the can. As we know from the kinematic equations,
the speed of the bullet can change quite handsomely in a short amount of time. Pointing the gun closer to the can mitigated this 
source of error.
\end{itemize}

\double\header{Additionally...}

The data we collected is incredibly accurate. For the kinematics approach, we had two observers watching the bullet so
they could accurately see where it landed. We also had people working together to guarantee that the gun was perfectly
horizontal each time. For the momentum approach, the trials were recorded in high definition, high speed video so that we could
accurately pinpoint the can's height. Adding to our accuracy, our trials were very consistent between each other. We did not
have to redesign our investigations part of the way through.\double

For the momentum approach, we did not control for the vertical orientation of the gun or the distance between the gun and the
can. Those measurements did not come up in the final calculations, so we assumed that they had no effect.\double

In the future, we could compare our results to repetitions of our investigation process or to officially published muzzle
speeds by the manufacturer.

\double\bigheader{Image Sources}

\begin{itemize}
\item Stool \\
https://target.scene7.com/is/image/Target/10165657\_Alt01?wid=520\&hei=520\\
\&fmt=pjpeg
\item Soda can \\
https://d30y9cdsu7xlg0.cloudfront.net/png/631937-200.png
\item Suction cup Nerf bullet \\
http://www.hhmdevelopments.co.uk/images/category\_17/Kids\%20NERF\%20Blaster\\
\%20Toys\%20100Pcs\%2010\%20Colors\%20Refill\%20Bullets\%20Darts\%20EVA\%20Foam\\
\%20with\%20Suction\%20Cup\%20for\%20NStrike\%20Elite\%20Kid\%20Toy\%20Gun\\
\%20Rifle\%20Blasters\%20Darts\%20SWSTTYLC.jpg
\item Nerf gun \\
http://i.ebayimg.com/00/s/MzMxWDUwMA==/z/2x4AAOSwVupTlwXL/\$\_32.JPG
\item Nerf bullet \\
https://images-na.ssl-images-amazon.com/images/I/51qAcBiOMfL.\_SL1010\_.jpg
\end{itemize}
\end{document}
